\documentclass[12pt]{article}
\usepackage[T2A]{fontenc}
\usepackage[utf8]{inputenc}
\usepackage[russian]{babel}
\usepackage{amsmath, amssymb}
\usepackage{geometry}
\usepackage{booktabs}
\usepackage{array}
\usepackage{multirow}
\usepackage{siunitx}
\usepackage{graphicx}
\usepackage{caption}
\usepackage{setspace}
\usepackage{hyperref}

\geometry{a4paper, left=25mm, right=15mm, top=20mm, bottom=20mm}
\onehalfspacing
\frenchspacing

\title{Индустриальное приложение: Система моделирования кристаллических решёток с дефектами}
\author{Попов Владимир, Б01-411}
\date{}

\begin{document}

\maketitle

\section{Постановка решаемой задачи}

\textbf{Основная цель:} Прогнозирование стабильных конфигураций кристаллических решёток материалов при наличии точечных и линейных дефектов под воздействием внешних условий (температура, механическое напряжение, радиационное облучение). Будем рассматривать решение для небольшой исследовательской группы из 10 человек в каком-нибудь НИИ.

\textbf{Конкретные подзадачи:}
\begin{itemize}
    \item Генерация начальных конфигураций решётки с заданным типом и концентрацией дефектов
    \item Минимизация полной потенциальной энергии системы
    \item Расчёт макроскопических свойств: модули упругости, энергия образования дефекта и т.д.
    \item Визуализация эволюции структуры
    \item Хранение и сравнение результатов для различных материалов и условий
\end{itemize}

\textbf{Актуальность:} Ускорение разработки радиационно-стойких материалов для ядерной энергетики и космической техники; оптимизация легирования полупроводников; исследование свойств новых материалов.

За основу взят \href{https://github.com/kzueirf12345/Programm_IpoF}{этот} проект по моделированию кристаллических решёток.

\section{Размерности данных}

\begin{table}[h]
\centering
\caption{Оценка объёмов данных}
\label{tab:data_sizes}
\begin{tabular}{p{0.35\textwidth}p{0.25\textwidth}p{0.35\textwidth}}
\toprule
\textbf{Тип данных} & \textbf{Оценка объёма} & \textbf{Пояснение} \\
\midrule
    Одна атомная конфигурация суперячейки & 
    \SI{0.5}{\mega\byte}--\SI{5}{\mega\byte} & 
    Координаты ($3 \times \text{float64}$) + тип атома. Рассматриваем около $10^4$--$10^6$, в среднем столько можно за адекватное время смоделировать. \\
\midrule
    Траектория релаксации (100 шагов) & 
    \SI{50}{\mega\byte}--\SI{500}{\mega\byte} & 
    Последовательность конфигураций для анимации \\
\midrule
    База данных (активная) & 
    \SI{2}{\tera\byte}--\SI{5}{\tera\byte} & 
    При 500--1000 конфигурациях на пользователя в год $\times$ 10 человек \\
\midrule
    Архив результатов & 
    \SI{10}{\tera\byte}--\SI{20}{\tera\byte} & 
    Долгосрочное хранение за 3--5 лет \\
\bottomrule
\end{tabular}
\end{table}

\noindent
Большую часть объёма занимают конфигурации и траектории. Метаданные: материал, тип дефекта, параметры расчёта — очень мало.

\section{Используемые вычислительные методы}

\subsection{Потенциал взаимодействия}

Для благородных газов и простых металлов применяется потенциал Леннарда-Джонса:
\begin{equation}
    V(r) = 4\varepsilon \left[ \left(\frac{\sigma}{r}\right)^{12} - \left(\frac{\sigma}{r}\right)^6 \right],
    \label{eq:lennard-jones}
\end{equation}
где $\varepsilon$ — глубина потенциальной ямы, $\sigma$ — расстояние, на котором $V(r)=0$.

Для ионных кристаллов используется потенциал Ми:
\begin{equation}
    V(r) = \frac{A}{r^n} - \frac{B}{r^m}, \quad n > m,
    \label{eq:mie}
\end{equation}
где первый член описывает отталкивание, второй — притяжение.

\subsection{Молекулярная динамика}

Моделирование движения атомов во времени путём численного решения уравнений Ньютона:
\begin{equation}
    m_i \frac{d^2 \mathbf{r}_i}{dt^2} = \mathbf{F}_i = -\nabla_{\mathbf{r}_i} E.
    \label{eq:newton}
\end{equation}
Интегрирование выполняется алгоритмом Верле:
\begin{equation}
    \mathbf{r}_i(t+\Delta t) = 2\mathbf{r}_i(t) - \mathbf{r}_i(t-\Delta t) + \frac{\mathbf{F}_i(t)}{m_i} \Delta t^2.
    \label{eq:verlet}
\end{equation}
Подходит для моделирования тепловых эффектов. Нужно бы ещё внести модель термостата.

Чтобы не считать за квадрат, можно ограничить радиус, в котором имеет смысл производить расчёты, за которым потенциал близок к нулю.

\section{Используемое программное и аппаратное обеспечение}
\label{sec:infrastructure}

\subsection{Программная архитектура}

\begin{table}[h]
\centering
\caption{Компоненты программной архитектуры}
\label{tab:software}
\begin{tabular}{p{0.25\textwidth}p{0.35\textwidth}p{0.35\textwidth}}
\toprule
\textbf{Компонент} & \textbf{Реализация} & \textbf{Пояснения} \\
\midrule
    GUI & 
    PyQt5 + OpenGL & 
    PyQt5 предоставляет виджеты интерфейса; 3D-рендеринг выполняется на GPU через модуль \texttt{QtOpenGL} \\
\midrule
    Бэкенд & 
    Python~3 + C++ (OpenMP) & 
    Управление логикой и БД — на Python; тяжёлые расчёты (силы, градиенты) — на C++ с параллелизацией через OpenMP (на CPU, потому что атомов не так много в районе 10^6, накладные расходы на представление данных GPU, наверное, будут больше) \\
\midrule
    База данных & 
    PostgreSQL & 
    Интеграция с Python (\texttt{psycopg2}) и C++ (\texttt{libpqxx}); хранение метаданных и путей к бинарным файлам \\
\bottomrule
\end{tabular}
\end{table}

\subsection{Аппаратная платформа}

\begin{table}[h]
\centering
\caption{Конфигурация аппаратной платформы}
\label{tab:hardware}
\begin{tabular}{p{0.3\textwidth}p{0.3\textwidth}p{0.35\textwidth}}
\toprule
\textbf{Компонент} & \textbf{Конфигурация} & \textbf{Пояснения} \\
\midrule
    Вычислительный сервер & 
    4 CPU на $\sim 28-32$ ядра, 4 NVIDIA GPU \SI{512}{\giga\byte} RAM & 
    Централизованный сервер для 5--10 одновременных пользователей; GPU используются для рендеринга, CPU - для рассчётов \\
\midrule
    Хранилище & 
    Что-то вместимостью порядка \SI{50}{\tera\byte} и не быстрое, и что-то вместимостью \SI{10}{\tera\byte} побыстрее.  & 
    База данных активных и архивных вычислений и вообще склад всякого.\\
\midrule
    Клиентские станции & 
    \SI{16}{\giga\byte} RAM, GPU с поддержкой OpenGL & 
    Достаточно для отображения GUI; тяжёлые расчёты выполняются на сервере \\
\bottomrule
\end{tabular}
\end{table}

\section{Полученные результаты}

Разработана целостная программно-аппаратная платформа для исследовательской группы по материаловедению около 10 человек, обеспечивающая:

\begin{enumerate}
    \item Моделирование эволюции кристаллических дефектов в системах из $10^4$--$10^6$ атомов с использованием молекулярной динамики на основе потенциалов Леннарда-Джонса и Ми.
    
    \item Совместную работу через централизованную архитектуру: сервер с многоядерными CPU и GPU обрабатывает расчёты и рендеринг; данные хранятся в СУБД PostgreSQL.
    
    \item Графический интерфейс на базе PyQt5 с аппаратно ускоренной 3D-визуализацией через OpenGL.
\end{enumerate}

\end{document}