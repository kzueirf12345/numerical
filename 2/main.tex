\documentclass[12pt]{article}
\usepackage[T2A]{fontenc}
\usepackage[utf8]{inputenc}
\usepackage[russian]{babel}
\usepackage{amsmath, amssymb}
\usepackage{geometry}
\usepackage{booktabs}
\usepackage{array}
\usepackage{multirow}
\usepackage{siunitx}
\usepackage{graphicx}
\usepackage{caption}
\usepackage{setspace}

\geometry{a4paper, left=25mm, right=15mm, top=20mm, bottom=20mm}
\onehalfspacing
\frenchspacing

\title{Бенчмаркинг для системы моделирования кристаллических решёток}
\author{Попов Владимир, Б01-411}
\date{}

\begin{document}

\maketitle

\section{Классы нагрузок и подходящие бенчмарки}

В представленном ранее проекте требуется определить, какое оборудование установить на сервере и на лабораторных машинах. Это можно решить, изучив соответствующие бенчмарки.

\begin{table}[h]
\centering
\caption{План бенчмаркинга инфраструктуры}
\label{tab:plan}
\begin{tabular}{p{0.25\textwidth}p{0.35\textwidth}p{0.35\textwidth}}
\toprule
\textbf{Компонент} & \textbf{Бенчмарк} & \textbf{Цель} \\
\midrule
    CPU (Вычисления) & 
    SPECrate Floating Point Base & 
    Оценка многопоточной производительности для вычислений с плавающей точкой.  \\
\midrule
    GPU (Визуализация) & 
    SPECviewperf & 
    Гарантирует, что выбранные GPU справятся с рендерингом в PyQt5+OpenGL миллионов атомов без задержек \\
\midrule
    Storage & 
    FIO, STREAM & 
    Оценка скорости ввода/вывода локальных конфигураций на лабораторном железе. Оценка пропускной способности памяти и работы БД, так как данные будут подгружаться с удалённого сервера. \\
\bottomrule
\end{tabular}
\end{table}

\section{Решение проблем бенчмаркинга}

Синтетические тесты могут не отражать реальную нагрузку молекулярной динамики, которая зависит от кэш-памяти при построении списков соседей и имеет специфический паттерн доступа к памяти.

SPECrate FP Включает реальные научные приложения, в том числе задачи вычислительной химии и физики, и SPECviewperf включает тесты визуализации научных данных.\

STREAM использует массивы размером более чем в 4 раза превышающие размер кэша последнего уровня, что соотносится с нашим случаем, ведь подгружать данные хочется максимально большими блоками, и параллелизация будет происходить на кубы непомещающиеся в кэш. При этом STREAM использует последовательный доступ к памяти, что соответствует обходу атомов в суперячейке, а в FIO можно протестировать случайный доступ, что соответствует выборке конфигураций из БД.


В итоге 
\begin{itemize}
    \item \textbf{SPECrate}: CPU, компилятор, библиотеки, ОС
    \item \textbf{SPECviewperf}: GPU, драйверы OpenGL, оконная система
    \item \textbf{FIO}: дисковая подсистема, СУБД, файловая система
    \item \textbf{STREAM}: CPU, контроллер памяти
\end{itemize}


\end{document}